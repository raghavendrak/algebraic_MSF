 \documentclass[11pt]{article}
 \usepackage[margin=1in,footskip=0.25in]{geometry}

\usepackage{braket,amsfonts}
\usepackage[english]{babel}
\usepackage[utf8]{inputenc}
\usepackage{amsmath}
\usepackage{graphicx}
\usepackage{amssymb}
% \usepackage{amsthm}
\usepackage{mathrsfs}
\usepackage[colorinlistoftodos]{todonotes}
\usepackage{enumitem}
\usepackage{yfonts}
% \usepackage{algorithm}
% \usepackage[linesnumbered,ruled]{algorithm2e}
% \usepackage[noend]{algpseudocode}
\usepackage{algorithmic}
\usepackage{cite}

\newcommand{\dd}[1]{\mathrm{d}#1}
\usepackage{filecontents}
\usepackage{ulem}
% to order citations
% \usepackage{cite}
% to allow clickable references
% \usepackage[hidelinks]{hyperref}
% for \coloneqq
\usepackage{mathtools}
\usepackage{adjustbox}
\usepackage{lipsum}% example text
\usepackage[section]{placeins}
\DeclareMathOperator*{\vecn}{vec}
\usepackage{bm}
% \usepackage{subfig}
\usepackage{graphicx,epstopdf}% <- Preamble
\usepackage[caption=false]{subfig}% <- Preamble


\title{}

\author{}
\newtheorem{thm}{Theorem}[section]
\newtheorem{lem}[thm]{Lemma}

\newtheorem{defn}[thm]{Definition}
\newtheorem{eg}[thm]{Example}
\newtheorem{ex}[thm]{Exercise}
\newtheorem{conj}[thm]{Conjecture}
\newtheorem{cor}[thm]{Corollary}
\newtheorem{claim}[thm]{Claim}
\newtheorem{rmk}[thm]{Remark}

\newcommand{\ie}{\emph{i.e.} }
\newcommand{\cf}{\emph{cf.} }
\newcommand{\into}{\hookrightarrow}
\newcommand{\dirac}{\slashed{\partial}}
\newcommand{\R}{\mathbb{R}}
\newcommand{\C}{\mathbb{C}}
\newcommand{\Z}{\mathbb{Z}}
\newcommand{\N}{\mathbb{N}}
\newcommand{\LieT}{\mathfrak{t}}
\newcommand{\T}{T}
\newcommand{\tree}[1]{{\mathcal{#1}}}
\newcommand{\tsr}[1]{\pmb{\mathcal{#1}}}
\newcommand{\fvcr}[1]{\bm{#1}}
\newcommand{\vcr}[1]{\mathbf{#1}}
\newcommand{\mat}[1]{\mathbf{#1}}
%\newcommand{\defeq}{\coloneqq}
\newcommand{\defeq}{=}
%\newcommand{\tnrm}[1]{\lvert \lvert #1 \rvert \rvert_F}
\newcommand{\tnrm}[1]{{\| #1 \|}_2}
\newcommand{\fnrm}[1]{{\| #1 \|}_F}
%\newcommand{\tinf}[1]{\left|\sigma_\text{min}(#1)\right|}
\newcommand{\tinf}[1]{\inf\{\vnrm{\fvcr{f}_{#1}}\}}
%\newcommand{\vnrm}[1]{{\lvert \lvert #1 \rvert \rvert}_2}
\newcommand{\vnrm}[1]{{\| #1 \|}_2}
%\newcommand{\vnrm}[1]{\| #1 \|_2}
\newcommand{\inti}[2]{\{{#1},\ldots, {#2}\}}
\newcommand{\M}{M}
\newcommand\bigzero{\makebox(0,0){\text{\huge0}}}
\newcommand{\es}[1]{{\color{red} #1}}



\begin{document}
\maketitle

Let $P^{(0)}$ be an upper triangular parent matrix, with $P^{(0)}_{ij} = 1$ if $\pi(i)=j$, i.e. $j$ is the parent of vertex $i$.
This matrix will have a single nonzero entry per row and represents a forest.
Given a set of undirected edges $E$, define upper triangular adjacency matrix $A$, such that $A_{ij}=1$ if $(i,j)\in E$ with $i<j$.
We seek to find a parent matrix which corresponds to a forest in which each tree is a connected component in $G=(V,E)$.
Let $A^{(0)}=A$.
Initialize, $\pi(i)=\max(i,\max_{j} A_{ij}j)$ and $P^{(0)}$ accordingly.
%\[P^{(0)}_{ij} = \begin{cases} i & \text{if} \forall (i,j)\in E\]
Compute
\begin{align*}
B^{(k)}_{ij} &= \max_{l} P^{(m)}_{li} A^{(k)}_{lj}  \\
A^{(k)}_{ij} &= \begin{cases} 0 &: i>j \\ \max(B^{(k)}_{ij},B^{(k)}_{ji}) &: i\leq j \end{cases}.
%A^{(k+1)}_{ij} &= \max_{l,m} P^{(k)}_{li} A^{(k)}_{lm} P^{(k)}_{mj}
%               &= \max_{m} A^{(k)}_{\pi(i),m} P^{(k)}_{mj}
%               &= \max_{(l,m)\in E^{(k)}} \pi(l)
%P^{(k+1)}_{ij} 
\end{align*}
So that if $A_{ij}^{(k)}=1$, then $B_{\pi(i),j}^{(k)}=1$, and $A_{\pi(i),j}^{(k)}=1$ if $\pi(i)<j$ or $A_{j,\pi(i)}^{(k)}=1$ if $\pi(i)\geq j$.
Execute until convergence for some $k$.
Then, set $\pi(i)=\max(i,\max_{j} A^{(k)}_{ij}j)$ and $P^{(m+1)}$ accordingly.


\end{document}
